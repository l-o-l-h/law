% Created 2025-11-17 Mon 16:41
% Intended LaTeX compiler: pdflatex
\documentclass[11pt]{article}
\usepackage[utf8]{inputenc}
\usepackage[T1]{fontenc}
\usepackage{graphicx}
\usepackage{longtable}
\usepackage{wrapfig}
\usepackage{rotating}
\usepackage[normalem]{ulem}
\usepackage{amsmath}
\usepackage{amssymb}
\usepackage{capt-of}
\usepackage{hyperref}
\author{LOLH}
\date{\textit{{[}2025-11-17 Mon 14:33]}}
\title{Fraud Claims with Breach of Confidential Relationship}
\hypersetup{
 pdfauthor={LOLH},
 pdftitle={Fraud Claims with Breach of Confidential Relationship},
 pdfkeywords={},
 pdfsubject={},
 pdfcreator={Emacs 30.2 (Org mode 9.7.11)}, 
 pdflang={English}}
\begin{document}

\maketitle
\tableofcontents

\section*{Summary}
\label{sec:orgadcf906}

In Washington State, fraud claims  are subject to a three-year statute
of  limitations  under  RCW  4.16.080(4), with  the  cause  of  action
accruing only upon  discovery of the facts constituting  fraud.  WA ST
4.16.080. However,  when fraud  occurs in a  confidential relationship
involving a transaction  not supported by full  consideration, and the
fraudulent party delays enforcement of a lease agreement for six years
without  seeking  rent, multiple  doctrines  may  extend or  toll  the
limitations period.  The existence  of a confidential relationship can
support   equitable   estoppel   if    the   fraudulent   party   made
representations that lulled the  victim into delayed action. \uline{Anh-Duong
Thi Ho v. Bach}, Not Reported in Pac. Rptr. (2018). \uline{Peterson v. Groves},
111  Wash.App. 306  (2002). Additionally,  where a  fiduciary duty  to
disclose  exists,   silence  about   material  facts   can  constitute
fraudulent  concealment that  tolls the  statute of  limitations until
discovery.  [KeyCite Yellow Flag] \uline{Crisman  v. Crisman}, 85 Wash.App. 15
(1997). The  delayed enforcement of the  lease agreement, particularly
when combined with the confidential  relationship and lack of adequate
consideration,  may  provide  grounds  for  arguing  either  equitable
estoppel or  fraudulent concealment,  though the plaintiff  must still
demonstrate  they   could  not  have  discovered   the  fraud  through
reasonable diligence.
\section*{Statutory Framework and Basic Discovery Rule}
\label{sec:orgb7285b6}

Washington's  fraud   statute  of  limitations  is   codified  in  RCW
4.16.080(4),  which provides  that fraud  actions "shall  be commenced
within three years"  but "the cause of  action in such case  not to be
deemed to have  accrued until the discovery by the  aggrieved party of
the facts constituting the fraud".  WA ST 4.16.080. This discovery rule
represents a departure from the  general rule that limitations periods
begin running when the wrongful act occurs, recognizing the inherently
concealed nature of fraudulent conduct.

Even  in  confidential  relationships,   the  burden  remains  on  the
plaintiff to demonstrate  that the fraud was  not discoverable through
due  diligence.   As  the  Court of  Appeals  emphasized  in  \uline{Douglass
v. Stanger}, "[e]ven in an action  for fraud where a fiduciary relation
exists,  the burden  is  upon the  plaintiff to  show  that the  facts
constituting  the  fraud  were  not   discovered  or  could  not  [be]
discovered  until within  3 years  prior  to the  commencement of  the
action."  [KeyCite Yellow Flag] \uline{Douglass v. Stanger}, 101 Wash.App. 243
(2000). This principle was reaffirmed in \uline{Sheehan v. Sheehan}, where the
court   held  that   "assuming,  without   deciding,  a   confidential
relationship existed between Mae and  Frank, this relationship did not
affect  when the  statute of  limitations  for Mae's  fraud and  undue
influence claims  accrued." \uline{Sheehan v.  Sheehan}, Not Reported  in P.3d
(2009).
\section*{Equitable Estoppel in Confidential Relationships}
\label{sec:org5c2a62e}

Washington  courts  recognize   that  confidential  relationships  can
support equitable estoppel to bar statute of limitations defenses when
the defendant's conduct  lulls the plaintiff into  delayed action. The
Court of  Appeals in \uline{nh-Duong  Thi Ho  v.  Bach} established  that "the
gravamen  of  equitable  estoppel  with  respect  to  the  statute  of
limitations is that the defendant  made representations or promises to
perform  which  lulled  the  plaintiff into  delayed  timely  action."
\uline{Anh-Duong Thi Ho v. Bach}, Not Reported in Pac. Rptr. (2018). The court
found  that  equitable  estoppel  barred the  statute  of  limitations
defense because  of the confidential relationship  between the parties
and the defendant's repeated promises to return property.

The elements of equitable estoppel  are: "(1) an admission, statement,
or act  inconsistent with  a claim afterward  asserted; (2)  action by
another in reasonable  reliance on that act,  statement, or admission;
and (3) injury to  the party who relied if the  court allows the first
party  to  contradict  or  repudiate  the  prior  act,  statement,  or
admission"  \uline{Anh-Duong Thi  Ho v.   Bach}, Not  Reported in  Pac.  Rptr.
(2018). Importantly, "the existence  of a confidential relationship is
a key indicator  of reliance." \uline{Anh-Duong Thi Ho v}.  Bach, Not Reported
in Pac. Rptr. (2018).

Peterson v. Groves provides  additional guidance on applying equitable
estoppel  in confidential  relationships  involving  promises to  pay.
\uline{Peterson v.  Groves},  111 Wash.App. 306 (2002).  The  court noted that
"the case for  estoppel is strengthened by the fact  that a promise to
pay has been  related to the happening of a  specific event" and found
that "the evidence  of a confidential relationship,  combined with the
evidence of Groves'  promises to pay upon the happening  of a specific
event, namely the sale of the  land, is sufficient to overcome summary
judgment on the issue of  equitable estoppel." \uline{Peterson v. Groves}, 111
Wash.App. 306 (2002). However, the court emphasized that estoppel does
not last forever and "the plaintiff  must act within a reasonable time
after discovering that  the promises relied on  were false." \uline{Peterson
v. Groves}, 111 Wash.App. 306 (2002).
\section*{Fraudulent Concealment and Fiduciary Duties}
\label{sec:orgb76d1d6}

Where  a  confidential  relationship   creates  a  fiduciary  duty  to
disclose,  silence  about  material facts  can  constitute  fraudulent
concealment  that  tolls the  statute  of  limitations. The  Court  of
Appeals in  \uline{Crisman v. Crisman}  held that managers who  owed fiduciary
duties to a  store owner "owed her an affirmative  duty of disclosure"
and    their   "silence    constitutes   an    affirmative   act    of
misrepresentation.   Consequently,  RCW   4.16.080(4),  the  statutory
discovery  rule for  fraud, applies."   [KeyCite Yellow  Flag] \uline{Crisman
v. Crisman}, 85  Wash.App. 15 (1997). The court explained  that "when a
duty to  disclose does exist,  however, the suppression of  a material
fact  is tantamount  to an  affirmative misrepresentation."   [KeyCite
Yellow Flag] \uline{Crisman v. Crisman}, 85 Wash.App. 15 (1997).

In \uline{Aug. v. U.S. Bancorp},  the court applied the fraudulent concealment
test from  \uline{Nordhorn v. Ladish  Co.}, which requires "(1)  the plaintiff
exercised due  diligence in trying to  uncover the facts, and  (2) the
defendant engaged in affirmative conduct  that would lead a reasonable
person to  believe that no  claim of fraudulent  concealment existed."
\uline{August v.   U.S. Bancorp},  146 Wash.App.   328 (2008).   This standard
prevents application of  the doctrine to cases  involving only passive
silence without an affirmative duty to disclose.
\section*{Inadequate Consideration in Confidential Relationships}
\label{sec:org4506ccb}

The lack of adequate consideration  in transactions between parties in
confidential  relationships can  be  relevant to  both the  underlying
fraud  claim  and the  statute  of  limitations analysis.   In  \uline{Lewis
v. Estate  of Lewis}, the court  noted that "two situations  lead to a
presumption of fraudulence:
\begin{itemize}
\item 1) when the consideration is so grossly inadequate as to shock the
conscience of the court" and

\item 2) when highly unreasonable consideration is coupled with other
inequitable incidents.
\end{itemize}

{[}KeyCite Yellow  Flag] \uline{Lewis  v.  Estate of  Lewis}, 45  Wash.App.  387
(1986). The \uline{Pedersen v. Bibioff} court held that "fraud may be presumed
in equity where the donor and donee share a confidential relationship"
when faced with "a highly  suspect transaction between persons sharing
a   confidential  relationship."   [KeyCite   Yellow  Flag]   \uline{Pedersen
v. Bibioff}, 64 Wash.App. 710 (1992).

\uline{Strong v.  Clark} demonstrates  how inadequate consideration can affect
the discovery rule  analysis.  [KeyCite Yellow Flag]  \uline{Strong v. Clark},
56 Wash.2d 230 (1960). The court  found that creditors were "deemed in
law to  have discovered on March  20, 1952, the alleged  inadequacy of
the consideration and 'the facts constituting the fraud'" when a lease
with option to purchase was recorded, providing constructive notice of
the  allegedly  insufficient  consideration.   [KeyCite  Yellow  Flag]
\uline{Strong v. Clark}, 56 Wash.2d 230 (1960).
\section*{Delayed Enforcement and Waiver Analysis}
\label{sec:org7eaf279}

The defendant's  failure to  enforce a lease  agreement for  six years
without  seeking  rent  presents   complex  issues  regarding  waiver,
modification, and potential  fraudulent concealment. Washington courts
distinguish between  waiver of contractual rights  and modification of
agreements.   In \uline{Panorama  Residential Protective  Ass'n v.   Panorama
Corp.},  the court  explained  that  "a waiver  is  an intentional  and
voluntary  relinquishment  of  a  known  right"  and  emphasized  that
"defendant's right to  collect the maximum rent under  the lease arose
each time an adjustment was called for by the lease."  [KeyCite Yellow
Flag]  \uline{Panorama Residential  Protective  Ass'n v.  Panorama Corp.},  28
Wash.App. 923 (1981).

\uline{Chabuk  v.    Miller}  provides  relevant  guidance   on  when  delayed
enforcement  constitutes waiver  of rent  collection rights.   \uline{Chabuk
v. Miller}, Not  Reported in Pac. Rptr.  (2021). The  court found that
where a landlord "never collected  rent from Miller, despite the lease
stated a rental amount," this  "supports the conclusion that he waived
that covenant"  and that "after years  of not collecting rent,  it was
unjust for Chabuk  to attempt to reinstate his rights  to collect rent
from  Miller without  first  giving her  a  reasonable opportunity  to
comply." \uline{Chabuk v. Miller}, Not Reported in Pac. Rptr. (2021).

However,  delayed  enforcement  alone may  not  constitute  fraudulent
concealment absent additional misleading  conduct. In \uline{Schreiner Farms,
Inc.   v.   American Tower,  Inc.},  the  court rejected  a  fraudulent
concealment  argument  where  the plaintiff  alleged  that  defendants
"disguised" a  sublease, finding that  this "does not  show fraudulent
concealment  because it  fails to  identify whether  the arrangement's
label is a  material fact, and whether Respondents  had an affirmative
duty  to use  a particular  label."  [KeyCite  Yellow Flag]  \uline{Schreiner
Farms, Inc. v. American Tower, Inc.}, 173 Wash.App. 154 (2013).
\section*{Application to the Specific Facts}
\label{sec:orgee611da}

The  combination   of  factors   present  in   this  case—confidential
relationship,  transaction lacking  full  consideration, and  six-year
delay in lease enforcement—creates potential grounds for extending the
statute of limitations through either equitable estoppel or fraudulent
concealment theories. The confidential  relationship may establish the
foundation for  arguing that  the defendant's silence  about enforcing
the lease, combined with the lack of consideration, created reasonable
reliance that prevented timely discovery of the fraud.

The six-year delay in seeking  rent could support an estoppel argument
if the defendant  made representations during this  period that lulled
the plaintiff  into believing no  enforcement would occur,  similar to
the repeated  promises in \uline{Anh-Duong  Thi Ho  v. Bach}, Not  Reported in
Pac.  Rptr.  (2018).  Alternatively, if the  confidential relationship
created a  duty to  disclose the  intent to enforce  the lease  or the
fraudulent nature of the original transaction, the defendant's silence
during  the six-year  period could  constitute fraudulent  concealment
under  [KeyCite  Yellow Flag]  \uline{Crisman  v.  Crisman}, 85  Wash.App.  15
(1997).

The success  of either theory  would depend on demonstrating  that the
plaintiff exercised reasonable diligence but could not have discovered
the fraud  earlier due to  the defendant's conduct in  maintaining the
confidential relationship while concealing the enforcement intentions.
\subsection*{Arguments and Rebuttals}
\label{sec:orgde2a995}

\subsubsection*{Arguments for Extended/Tolled Limitations Period}
\label{sec:org0335fe5}

\begin{itemize}
\item Confidential Relationship Estoppel
\label{sec:orgd904b30}

The  existence  of  a  confidential  relationship  combined  with  the
defendant's six-year  delay in enforcing the  lease created reasonable
reliance that  no enforcement would  occur, similar to the  pattern in
\uline{Anh-Duong Thi Ho  v. Bach} where repeated promises  to perform estopped
the  defendant  from  raising  the  statute  of  limitations  defense.
\uline{Anh-Duong   Thi   Ho  v.    Bach},   Not   Reported  in   Pac.    Rptr.
(2018).
\item Anticipated Rebuttals
\label{sec:org815d9cd}

Defendants may  argue that  mere silence  or non-enforcement  does not
constitute  the  type  of  affirmative  representations  required  for
estoppel, and that the plaintiff  had constructive notice of the lease
terms regardless of enforcement patterns.
\item Fraudulent Concealment Through Fiduciary Silence
\label{sec:orge47fe17}

Where a confidential relationship creates fiduciary duties, the
defendant's failure to disclose intent to enforce the lease or the
fraudulent nature of the original transaction constitutes fraudulent
concealment that tolls the limitations period under [KeyCite Yellow
Flag] \uline{Crisman v. Crisman}, 85 Wash.App. 15 (1997).
\item Anticipated Rebuttals
\label{sec:org585f6de}

Defendants may  contend that  no specific  fiduciary duty  to disclose
enforcement intentions  existed, and that the  written lease agreement
provided adequate notice of potential enforcement rights.
\item Inadequate Consideration as Fraud Indicator
\label{sec:org1616919}

The transaction's lack  of full consideration, when  combined with the
confidential relationship,  creates circumstances  where fraud  may be
presumed under  Pedersen v.  Bibioff  and supports arguments  that the
fraudulent  nature  was  concealed   [KeyCite  Yellow  Flag]  \uline{Pedersen
v. Bibioff}, 64 Wash.App. 710 (1992).
\item Anticipated Rebuttals
\label{sec:org2e0435d}

Defendants may argue that any inadequacy of consideration was apparent
from the  transaction documents and that  sophisticated parties cannot
claim inability to discover obvious consideration deficiencies.
\end{itemize}
\subsubsection*{Arguments Against Extension}
\label{sec:org856c3b2}

\begin{itemize}
\item Due Diligence Burden Unaffected
\label{sec:org2b774e4}

Even in confidential relationships, the plaintiff bears the burden of
showing fraud was not discoverable through reasonable investigation,
as established in [KeyCite Yellow Flag] \uline{Douglass v. Stanger}, 101
Wash.App. 243 (2000) and \uline{Sheehan v. Sheehan}, Not Reported in P.3d
(2009).
\item Anticipated Rebuttals
\label{sec:orgb6423c7}

Plaintiffs  may  argue  that   the  confidential  relationship  itself
prevented reasonable  investigation and  that the  defendant's conduct
during the six-year period reinforced the concealment.
\item Insufficient Affirmative Concealment
\label{sec:org87fa59c}

Delayed  enforcement   alone  does  not  constitute   the  affirmative
concealment required  for tolling, as  shown in Schreiner  Farms where
alleged disguising of transaction terms was insufficient without proof
of  duty to  use  particular labels  [KeyCite  Yellow Flag]  \uline{Schreiner
Farms,   Inc.   v.   American   Tower,   Inc.},   173   Wash.App.   154
(2013).
\item Anticipated Rebuttals
\label{sec:org05b80f2}

Plaintiffs may distinguish their  case by emphasizing the confidential
relationship  created  specific  disclosure   duties  not  present  in
ordinary commercial relationships.
\item Constructive Notice from Documents
\label{sec:org59e3f32}

The existence of  the lease agreement provided  constructive notice of
the  defendant's   rights  and   the  transaction   terms,  preventing
application of discovery rule protection  similar to the public record
notice  in [KeyCite  Yellow  Flag]  \uline{Strong v.  Clark},  56 Wash.2d  230
(1960).
\item Anticipated Rebuttals
\label{sec:orgc47448e}

Plaintiffs may argue that the confidential relationship and pattern of
non-enforcement created reasonable belief that the written terms would
not  be enforced,  distinguishing from  cases involving  purely public
record notice.
\end{itemize}
\subsubsection*{Cases on Both Sides}
\label{sec:org625bc1e}

\begin{itemize}
\item Cases Supporting Extended/Tolled Limitations Period
\label{sec:org61216a8}

\uline{Anh-Duong Thi  Ho v.  Bach}, Not  Reported in Pac.  Rptr. (2018)  — The
court held that  equitable estoppel barred the  statute of limitations
defense due to the  confidential relationship and defendant's repeated
promises to  return property.  The court  found that  the grandparents
reasonably  relied  on  Bach's  promises  due  to  their  confidential
relationship, causing them to delay filing suit.

\uline{Peterson v.  Groves}, 111 Wash.App. 306  (2002) — The court  ruled that
evidence of a confidential relationship  combined with promises to pay
upon a specific  event was sufficient to overcome  summary judgment on
equitable   estoppel.   The   court   emphasized   that   confidential
relationships create particularly strong grounds for reliance that can
support estoppel defenses.

{[}KeyCite Yellow Flag] \uline{Crisman v. Crisman}, 85 Wash.App. 15 (1997) — The
court  found that  managers' fiduciary  duties created  an affirmative
obligation  to disclose,  making their  silence constitute  fraudulent
concealment that  tolled the statute  of limitations.  The  court held
that  when  disclosure duties  exist,  suppression  of material  facts
equals affirmative misrepresentation.
\item Cases Against Extension
\label{sec:orgf3245ff}

{[}KeyCite Yellow Flag] \uline{Douglass v.  Stanger}, 101 Wash.App. 243 (2000) —
The court held  that even in fiduciary  relationships, plaintiffs must
prove fraud  was not discoverable  through due diligence  within three
years of suit. The court emphasized that confidential relationships do
not  eliminate the  basic burden  of showing  reasonable inability  to
discover fraud.

\uline{Sheehan v. Sheehan}, Not Reported in P.3d (2009) — The court ruled that
confidential relationships  do not affect when  statute of limitations
accrues  and  applied  the  general  discovery  rule  without  special
accommodation  for   the  family  relationship.  The   court  rejected
arguments that confidential relationships  create automatic tolling of
limitations periods.

{[}KeyCite Yellow Flag]  \uline{Schreiner Farms, Inc. v.  American Tower, Inc.},
173 Wash.App. 154 (2013) — The  court found that alleged disguising of
transaction terms was insufficient  for fraudulent concealment without
proof of  duty to  disclose specific  information. The  court required
clear   evidence  of   affirmative   concealment   rather  than   mere
non-disclosure or delayed enforcement.
\end{itemize}
\subsection*{Practical Implications}
\label{sec:org3dfd65c}

The  Washington   framework  for  fraud  statute   of  limitations  in
confidential relationships creates  several practical implications for
litigants and practitioners. The discovery rule and equitable estoppel
doctrines  provide meaningful  protection  for  those in  confidential
relationships who may  be prevented from timely  discovering fraud due
to trust  and reliance,  but these  protections come  with significant
evidentiary burdens.

Plaintiffs must develop evidence  showing lack of reasonable discovery
opportunities,  which can  be difficult  when documentary  evidence of
concealment may be  controlled by the allegedly  fraudulent party. The
requirement to act with due  diligence once estoppel circumstances end
means victims cannot indefinitely delay filing suit, creating pressure
for prompt legal consultation after discovery of potential fraud.

Evidence  preservation  becomes  critical for  establishing  both  the
confidential   relationship   and   the  grounds   for   estoppel   or
concealment. The importance  of documenting promises, representations,
and  relationship dynamics  cannot  be overstated,  as these  elements
often determine whether extended limitations periods apply.

Defendants  may  argue  that   delayed  enforcement  constituted  mere
forbearance  rather  than  fraudulent concealment,  requiring  careful
factual development about any affirmative representations or duties to
disclose. The complexity of proving  both the underlying fraud and the
estoppel or concealment theories  may make smaller claims economically
impractical to pursue, creating access to justice concerns for victims
of fraud in confidential relationships.
\section*{Recent Developments}
\label{sec:org5d23d33}

Recent Washington  developments have  both strengthened  and clarified
the  standards   for  extending   fraud  statute  of   limitations  in
confidential relationships. \uline{Anh-Duong Thi Ho v. Bach} (2018) reinforced
that  confidential relationships  combined with  repeated promises  to
perform can  create effective estoppel against  statute of limitations
defenses,  extending  protection  particularly for  vulnerable  family
members   who   may   be   exploited  by   those   in   positions   of
trust.  \uline{Anh-Duong  Thi Ho  v.   Bach},  Not  Reported in  Pac.  Rptr.
(2018).   This decision  demonstrates  continued judicial  recognition
that  confidential  relationships  require  special  consideration  in
limitations analysis.

\uline{Chabuk  v.   Miller}  (2021)  shows  courts  increasingly  scrutinizing
long-term   non-enforcement    patterns   in    lease   relationships,
distinguishing  between   mere  forbearance   and  actual   waiver  of
contractual   rights.    \uline{Chabuk   v.     Miller},   Not   Reported   in
Pac. Rptr. (2021). This development  suggests that extended periods of
non-enforcement  may   support  arguments  for  estoppel   or  waiver,
particularly when combined with other inequitable circumstances.

Recent decisions  like \uline{Schreiner Farms  v. American Tower}  (2013) have
tightened requirements  for proving fraudulent  concealment, requiring
clear   evidence  of   affirmative   concealment   rather  than   mere
non-disclosure.   [KeyCite  Yellow  Flag] \uline{Schreiner  Farms,  Inc.  v.
American Tower, Inc.}, 173 Wash.App.  154 (2013).  This trend suggests
courts  are  becoming  more  demanding   in  their  analysis  of  what
constitutes  sufficient  concealment   to  toll  limitations  periods,
potentially  making  it more  difficult  for  plaintiffs to  establish
fraudulent concealment theories.
\section*{Related Issues}
\label{sec:org111a768}

\begin{description}
\item[{Breach  of  Fiduciary  Duty  Claims}] Frequently  accompany  fraud
allegations  in confidential  relationships and  are subject  to the
same  three-year statute  of  limitations under  RCW 4.16.080,  with
similar discovery rule applications and burden of proof requirements
for demonstrating when the breach should have been discovered.

\item[{Undue  Influence  in   Confidential  Relationships }] Particularly
relevant when transactions lack  adequate consideration, with burden
shifting to the defendant to prove absence of undue influence once a
confidential relationship is  established, creating parallel grounds
for challenging transactions beyond fraud theories.

\item[{Consumer Protection Act Violations}] RCW 19.86 claims often overlap
with   fraud   in   commercial   contexts   involving   confidential
relationships,  with a  four-year limitations  period and  different
discovery  requirements that  may provide  alternative theories  for
recovery when fraud limitations periods have expired.

\item[{Constructive  Trust  and  Unjust Enrichment }] Equitable  remedies
frequently  sought  in  cases   involving  fraudulent  transfers  in
confidential  relationships,  with  different  limitations  analysis
under  equity principles  that may  extend beyond  traditional fraud
statute of limitations constraints.
\end{description}
\section*{Commentary on This Question}
\label{sec:org88a5f41}

The statute of limitations for fraud claims generally begins to run at
the time the plaintiff discovers  or reasonably should have discovered
the  fraud, rather  than  at  the time  of  the  fraudulent act.  This
discovery  rule   applies  broadly,   including  in   cases  involving
intentional  torts  and insurance  fraud.  Courts  apply a  reasonable
diligence  standard  to  determine  when  the  plaintiff  should  have
discovered the fraud, and knowledge or suspicion sufficient to trigger
a  duty  of  inquiry  can start  the  limitations  period.  Fraudulent
concealment by the  defendant may toll the statute if  active steps to
prevent discovery are  shown, though proof of  such concealment varies
by jurisdiction and claim type.
\begin{itemize}
\item AMLOT § 5:40,
\item 43 A.L.R.3d 429.
\end{itemize}

Specifically,  Washington   law  imposes   a  three-year   statute  of
limitations  on   claims  involving  securities  fraud   and  consumer
protection  claims tied  to  fraudulent misrepresentations.   Notably,
Washington  Rev.    Code  §  21.20.430(4)(b)  provides   a  three-year
limitation based on the sale date or discovery of fraud, complementing
this discovery-based rule. SECBLUE  § 9:97. Moreover, courts recognize
a  distinction  between the  statute  of  limitations and  statute  of
repose, with equitable doctrines like tolling applying primarily where
defendants have actively concealed  relevant facts. The discovery rule
and tolling principles aim to balance protecting defendants from stale
claims and allowing plaintiffs fair  opportunity to pursue claims once
fraud is uncovered.
\begin{itemize}
\item 32 AMJUR POF 3d 129,
\item SECBLUE § 9:97.
\end{itemize}
\section*{AI}
\label{sec:orga1c1de2}
The above response is AI-generated and may contain errors. It should be verified for accuracy.
\end{document}
