% Created 2025-11-17 Mon 14:21
% Intended LaTeX compiler: pdflatex
\documentclass[11pt]{article}
\usepackage[utf8]{inputenc}
\usepackage[T1]{fontenc}
\usepackage{graphicx}
\usepackage{longtable}
\usepackage{wrapfig}
\usepackage{rotating}
\usepackage[normalem]{ulem}
\usepackage{amsmath}
\usepackage{amssymb}
\usepackage{capt-of}
\usepackage{hyperref}
\author{LOLH}
\date{\textit{{[}2025-11-17 Mon 13:51]}}
\title{Breach of Confidential Relationship Causing Financial Harm}
\hypersetup{
 pdfauthor={LOLH},
 pdftitle={Breach of Confidential Relationship Causing Financial Harm},
 pdfkeywords={},
 pdfsubject={},
 pdfcreator={Emacs 30.2 (Org mode 9.7.11)}, 
 pdflang={English}}
\begin{document}

\maketitle
\tableofcontents

\section*{Summary}
\label{sec:org8079384}

Under Washington State  law, a claim for breach  of close confidential
relationship  that   causes  financial  harm  requires   proving  four
essential elements:

\begin{itemize}
\item 1) the existence of a confidential relationship where one person has
gained the confidence of another and purports to act or advise with
the other's interest in mind,

\item 2) breach of that relationship through bad faith conduct,

\item 3) resulting financial injury or  harm, and

\item 4) proximate  causation between  the  breach and  the damages.
\end{itemize}

Washington  courts have  established  that confidential  relationships
arise from  personal rather than professional  connections and require
more than  mere friendship  or family ties—there  must be  evidence of
special confidence reposed in advice and the advisor's acceptance of a
role to act in the principal's interest.

Financial  harm must  be  proven with  clear,  cogent, and  convincing
evidence.

Remedies may include
\begin{itemize}
\item constructive trust,
\item restitution,
\item disgorgement, or
\item monetary damages.
\end{itemize}
\section*{Legal Framework for Confidential Relationships}
\label{sec:org3bee969}

Washington courts  recognize that breach of  confidential relationship
claims are closely related to undue influence claims and are developed
through  judicial  precedent  rather than  statutory  authority.   The
foundational case  [KeyCite Yellow] \uline{Flag McCutcheon  v.  Brownfield}, 2
Wash.App.  348  (1970) established  the basic framework  for analyzing
confidential  relationships in  the  context of  undue influence  that
continues  to  govern  these  claims   today.   The  court  defined  a
confidential  relationship  as  existing  "when  one  has  gained  the
confidence of the other and purports to act or advise with the other's
interest in mind."  [KeyCite Yellow  Flag] \uline{McCutcheon v. Brownfield}, 2
Wash.App.  348  (1970).  This relationship is  "particularly likely to
exist where there is a family relationship," but family ties alone are
insufficient  to  establish  the necessary  legal  standard.  [KeyCite
Yellow Flag] \uline{McCutcheon v. Brownfield}, 2 Wash.App. 348 (1970).

The  Washington Court  of  Appeals in  [KeyCite  Yellow Flag]  \uline{Seattle
Northwest Securities Corp. v. SDG  Holding Co., Inc.}, 61 Wash.App. 725
(1991)  clarified that  confidential relationships  "may exist  either
because  of  the  nature  of  the  relationship  between  the  parties
historically    considered   fiduciary    in   character"    such   as
attorney-client   or   physician-patient    relationships,   "or   the
confidential relationship between persons involved may exist in fact."
{[}KeyCite  Yellow Flag]  \uline{Seattle  Northwest Securities  Corp.  v.   SDG
Holding  Co.,  Inc.}, 61  Wash.App.  725  (1991). This  distinction  is
crucial  because it  separates  formal fiduciary  duties arising  from
professional  relationships   from  confidential   relationships  that
develop through personal interactions and dependency.
\section*{Essential Elements of the Claim}
\label{sec:org68c7dd4}

\subsection*{Element One: Existence of Confidential Relationship}
\label{sec:org5755f78}

The   first  element   requires  establishing   that  a   confidential
relationship actually  existed between  the parties.   [KeyCite Yellow
Flag] \uline{Lewis v.  Estate of Lewis}, 45 Wash.App. 387  (1986) provides the
most  detailed  analysis  of   this  requirement,  holding  that  "the
essential elements  of a  confidential relationship  are
\begin{itemize}
\item (1) that the parent reposes some special confidence in the child's
advice and
\item (2) that the child purports to advise with his parent's interests in
mind."  [KeyCite Yellow Flag] \uline{Lewis v.  Estate of Lewis}, 45
Wash.App.  387 (1986).
\end{itemize}

The court  emphasized that while confidential  relationships are "more
likely to  exist between  parent and child,  parentage alone  does not
create  the  relationship"  and  "additional  factors  are  required."
{[}KeyCite  Yellow Flag]  \uline{Lewis v.  Estate  of Lewis},  45 Wash.App.  387
(1986).

The McCutcheon court explained that parentage may furnish the occasion
for a confidential relationship "when  the parent may become dependent
upon the  child, either for  support and  maintenance, or for  care or
protection in business matters as well, or for both, and the child, by
virtue  of  factors of  personality  and  superior knowledge  and  the
assumption of the role of adviser  accepted by the parent, may acquire
a status, vis-a-vis the parent, that will bring about the confidential
relationship."   [KeyCite Yellow  Flag]  \uline{McCutcheon  v. Brownfield},  2
Wash.App.   348 (1970).  This analysis  requires examining  the actual
dynamics  of  dependency  and   advice-giving  rather  than  presuming
relationships based on family status.

Recent cases have  reinforced that mere friendship  is insufficient to
establish a confidential relationship. In [KeyCite Yellow Flag] \uline{Kitsap
Bank  v.  Denley},  177  Wash.App.   559 (2013),  the  court held  that
"although Lanterno  was Correll's close  friend, a friendship,  on its
own, does not establish a confidential relationship."  [KeyCite Yellow
Flag] \uline{Kitsap  Bank v.   Denley}, 177 Wash.App.   559 (2013).  The court
distinguished confidential relationships,  which "generally arise from
personal  relationships," from  fiduciary relationships,  which "arise
from professional  relationships."  [KeyCite Yellow Flag]  \uline{Kitsap Bank
v. Denley}, 177 Wash.App. 559 (2013).
\subsection*{Element Two: Breach of the Confidential Relationship}
\label{sec:org8d830f2}

The  second   element  requires  proving   that  the  person   in  the
confidential relationship  breached their duty of  good faith. \uline{Georges
v.   Loutsis},  20   Wash.2d  92  (1944)  established   that  while  "a
confidential relationship demands  the best of good  faith in business
dealings." there  "must be a  breach of good  faith before a  cause of
action arises."  \uline{Georges  v.  Loutsis}, 20 Wash.2d 92  (1944) The court
held that the  confidential relationship alone "forms no  ground for a
cause  of  action" without  evidence  of  actual misconduct.   \uline{Georges
v. Loutsis}, 20 Wash.2d 92 (1944).

The  breach typically  involves the  person in  the position  of trust
using their influence  for personal gain at the expense  of the person
who reposed  confidence in  them. In the  context of  undue influence,
McCutcheon  identified  several  factors  that  can  evidence  breach,
including:
\begin{itemize}
\item taking advantage of the principal's impaired mental condition,
\item acting contrary to the principal's prior expressed intentions,
\item failing to provide independent advice, preparing documents through
the advisor's personal attorney without separate consultation, and
\item creating arrangements that leave the principal financially
dependent.
\end{itemize}

{[}KeyCite  Yellow Flag]  \uline{McCutcheon  v.  Brownfield},  2 Wash.App.   348
(1970).
\subsection*{Element Three: Financial Injury or Harm}
\label{sec:org44da48c}

The third element requires proof  of actual financial injury resulting
from  the  breach.  Recent  Washington  cases  have strengthened  this
requirement,  with \uline{Matter  of Estate  of Kolesar},  27 Wash.App.2d  166
(2023) emphasizing  that "in  cases where a  confidential relationship
resulted in undue influence, there  typically is evidence of some type
of loss resulting from that relationship" and "there must be something
more  than  the  relationship."   \uline{Matter  of  Estate  of  Kolesar},  27
Wash.App.2d 166  (2023). The court  explained that "a focus  on simply
what was given up, without consideration  of what was received, is not
sufficient to establish that  an unfair transaction occurred." \uline{Matter
of Estate of Kolesar}, 27 Wash.App.2d 166 (2023).

The financial  harm element  aligns with  general breach  of fiduciary
duty requirements.  [KeyCite Yellow Flag]  \uline{Arden v. Forsberg \& Umlauf,
P.S.}, 193 Wash.App. 731 (2016) held that "The plaintiff must prove

\begin{itemize}
\item 1) the existence  of a  fiduciary duty,

\item 2) a breach of that fiduciary duty,

\item 3) resulting injury, and

\item 4) that the breach of duty proximately caused the injury."  [KeyCite
Yellow Flag] \uline{Arden v.  Forsberg \& Umlauf, P.S.}, 193 Wash.App.  731
(2016).
\end{itemize}

The court emphasized  that "an attorney can be liable  for a breach of
fiduciary  duty only  if  the breach  caused  some injury."   [KeyCite
Yellow  Flag] \uline{Arden  v. Forsberg  \&  Umlauf, P.S.},  193 Wash.App.  731
(2016).
\subsection*{Element Four: Proximate Causation}
\label{sec:org59865bd}

The  fourth  element requires  establishing  that  the breach  of  the
confidential relationship proximately caused  the financial harm. This
element ensures  that damages  are not awarded  for losses  that would
have occurred regardless of the breach. The Arden court confirmed that
causation  is  an  essential   element  of  fiduciary  breach  claims,
requiring  proof  "that the  breach  of  duty proximately  caused  the
injury."  [KeyCite Yellow Flag] \uline{Arden  v. Forsberg \& Umlauf, P.S.}, 193
Wash.App. 731 (2016).
\section*{Burden of Proof and Evidentiary Standards}
\label{sec:orgc4a410f}

Washington   courts   apply   heightened  evidentiary   standards   to
transactions between persons in confidential relationships. \uline{McCutcheon}
held that "because  undue influence is treated in law  as a species of
fraud,  evidence  of   a  gift  between  persons   in  a  confidential
relationship must  be clear, cogent and  convincing."  [KeyCite Yellow
Flag] \uline{McCutcheon v.  Brownfield}, 2 Wash.App. 348 (1970). This elevated
burden  reflects  the  serious  nature  of  the  allegations  and  the
potential for abuse in relationships involving trust and dependency.

When a confidential  relationship is established, the  burden of proof
may shift in certain contexts.   [KeyCite Yellow Flag] \uline{Lewis v. Estate
of  Lewis}, 45  Wash.App.  387 (1986)  explained  that "generally,  one
seeking to set aside an inter vivos gift has the burden of showing the
invalidity thereof," but "the burden shifts, however, if the donor and
donee  shared a  confidential  relationship."   [KeyCite Yellow  Flag]
\uline{Lewis v. Estate of Lewis}, 45  Wash.App. 387 (1986). In such cases, "the
donee must then prove that a gift was intended and that it was not the
product of undue influence."  [KeyCite Yellow Flag] \uline{Lewis v. Estate of
Lewis}, 45 Wash.App. 387 (1986).
\section*{Available Remedies}
\label{sec:org98c6289}

Washington  courts provide  several  remedies for  proven breaches  of
confidential relationships.  \uline{Venwest Yachts, Inc. v.  Schweickert}, 142
Wash.App.   886 (2008)  explained  that "a  constructive  trust is  an
equitable  remedy  that  compels   restoration  where  a  party  gains
something for himself which, 'in equity and good conscience, he should
not be permitted to hold.'"  \uline{Venwest Yachts, Inc.  v. Schweickert}, 142
Wash.App. 886 (2008). The court emphasized that "in deciding to impose
a  constructive trust,  the  question is  whether  the enrichment  was
unjust, not whether  the holder of the property acted  with bad motive
or  malicious intent."   \uline{Venwest  Yachts, Inc.   v.  Schweickert},  142
Wash.App. 886 (2008).

Courts may also order disgorgement  of profits obtained through breach
of  confidential  relationships,  similar to  other  fiduciary  breach
remedies, though such awards require  proof of the breach and improper
gains. [NEED CITATIONS HERE]
\section*{Distinguishing Confidential from Fiduciary Relationships}
\label{sec:org4d322a1}

Washington  courts  carefully distinguish  confidential  relationships
from  formal fiduciary  relationships.  [KeyCite  Yellow Flag]  \uline{Kitsap
Bank v. Denley},  177 Wash.App. 559 (2013)  explained that confidential
relationships  "generally  arise  from personal  relationships"  while
fiduciary  relationships  "arise   from  professional  relationships."
{[}KeyCite Yellow Flag] \uline{Kitsap Bank v. Denley}, 177 Wash.App. 559 (2013).
This distinction affects both the analysis of whether the relationship
exists and the duties that flow from it.

The \uline{McCutcheon}  court noted that confidential  relationships may exist
either because of relationships  "historically considered fiduciary in
character"  such as  "trustee  and beneficiary,  principal and  agent,
partner and partner, husband and wife, physician and patient, attorney
and  client" or  because  "confidential  relationship between  persons
involved  may exist  in fact."   [KeyCite Yellow  Flag] \uline{McCutcheon  v.
Brownfield},  2  Wash.App.   348   (1970).   The  factual  confidential
relationships require case-by-case analysis of the actual relationship
dynamics.
\section*{Recent Developments}
\label{sec:org70ee826}

Recent  Washington  cases  show  important  developments  in  applying
confidential relationship principles. \uline{Matter  of Estate of Kolesar}, 27
Wash.App.2d 166 (2023) in 2023 emphasized that courts require evidence
of actual loss rather than  just unbalanced transactions, stating that
"there must  be something more  than the relationship" to  establish a
valid claim.  This represents a  trend toward requiring  more concrete
evidence of harm beyond the existence of the relationship itself.

\uline{Mueller v. Wells}, 185 Wash.2d 1 (2016) in 2016 refined the analysis to
focus  on whether  there  exists  "a level  of  trust  that leads  the
testator to believe that the beneficiary  is acting in his or her best
interests, creating an opportunity for  the beneficiary to exert undue
influence."  [KeyCite  Yellow Flag] \uline{Mueller  v.  Wells}, 185  Wash.2d 1
(2016).  This formulation  emphasizes  the  practical opportunity  for
abuse rather than abstract relationship categories.

The 2016 [KeyCite  Yellow Flag] \uline{Arden v. Forsberg \&  Umlauf, P.S.}, 193
Wash.App. 731 (2016) decision clarified  that breach of fiduciary duty
claims, which  share similar  elements with  confidential relationship
breach claims, require proof of actual resulting damages, not just the
breach  itself. This  reinforces the  trend toward  requiring concrete
proof  of  financial  harm  rather than  presuming  damages  from  the
relationship breach.
\section*{Related Issues}
\label{sec:org88e09f6}

\begin{description}
\item[{Undue  influence claims }] often  arise together  with confidential
relationship breach claims, particularly in estate and gift contexts
where the  confidential relationship creates a  presumption of undue
influence

\item[{Constructive trust claims}] requently sought as an equitable remedy
when  someone  breaches  a  confidential  relationship  and  obtains
property or benefits they should not retain in good conscience

\item[{Professional  malpractice}] when confidential  relationships arise
from professional services like attorney-client or physician-patient
relationships, creating overlapping duties and potential claims

\item[{Elder  abuse}] in cases  involving older  vulnerable adults  where
family members or caregivers  exploit confidential relationships for
financial gain, often triggering both civil and criminal liability
\end{description}
\section*{Commentary on This Question}
\label{sec:org79556d8}

A claim for breach of a confidential or fiduciary relationship causing
financial  harm generally  requires  showing

\begin{itemize}
\item 1)  the  existence of  a confidential or fiduciary relationship,

\item 2)  a breach or abuse of that duty, and

\item 3) resulting damages  proximately caused by the  breach.
\end{itemize}

A confidential  relationship arises when  one party reposes  trust and
confidence  in another  who  holds a  superior  or dominant  position,
resulting in a  disparity of bargaining power or  influence.

The duty entails
\begin{itemize}
\item acting with  loyalty and care,
\item avoiding self-dealing, and
\item not  profiting  from  the relationship  without  consent.
\end{itemize}

Courts examine  the totality  of circumstances  to establish  both the
relationship and  its breach;  mere breach  of oral  agreements absent
fraud, mistake, or confidential relationship is insufficient to impose
equitable remedies  such as constructive  trusts.
\begin{itemize}
\item 79 AMJUR POF  3d 269,
\item BOGERT § 482.
\end{itemize}


Under Washington law and  similarly in other jurisdictions, plaintiffs
must prove  damages resulting from  the breach and a  proximate causal
link  to the  misconduct. Constructive  fraud claims  may not  require
actual  intent to  deceive  but arise  from breach  of  duty within  a
relationship of  trust, inducing justifiable reliance  and injury. The
breach  can  be  inequitable  conduct,  self-dealing,  or  failure  of
loyalty,  often supported  by evidence  of dominance  or overmastering
influence.  The terms  fiduciary  and  confidential relationships  are
closely related,  with fiduciary duties often  implying disclosure and
loyalty obligations enforceable by law.
\begin{itemize}
\item 77 COA2d 215,
\item BOGERT § 482.
\end{itemize}


The above response  is AI-generated and may contain  errors. It should
be verified for accuracy.
\end{document}
